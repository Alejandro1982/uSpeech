The u\-Speech library provides an interface for voice recognition using the Arduino. It currently produces phonemes, often the library will produce junk phonemes. Please bare with it for the time being. A noise removal function is underway. \subsection*{Minimum Requirements}

The library is quite intensive on the processor. Each sample collection takes about 3.\-2 milliseconds so pay close attention to the time. The library has been tested on the Arduino Uno (A\-T\-Mega32). Each signal object uses up 160bytes. No real time scheduler should be used with this.

\subsection*{Features}


\begin{DoxyItemize}
\item Letter based recognition
\item Small memory footprint
\item Arduino Compatible
\item No training required
\item Fixed point arithmetic (not anymore)
\item 30\% -\/ 40\% accuracy if based on phonemes, up to 80\% if based on words.
\item Plugs directly into an {\ttfamily analog\-Read()} port
\end{DoxyItemize}

\subsection*{Documentation}

Head over to the \href{https://github.com/arjo129/uSpeech/wiki}{\tt wiki} and you will find most of the documentation required.

\subsection*{Algorithm}

The library utilizes a special algorithm to enable speech detection. First the complexity of the signal is determined by taking the absolute derivative of the signal multiplying it by a fixed point saclar and then dividing it by the absolute integral of the signal. Consonants (other than R,L,N and M) have a value above 40 and vowels have a value below 40. Consonants, they can be divided into frictaves and plosives. Plosives are like p or b whereas frictaves are like s or z. Generally each band of the complexity coeficient (abs derivative over abs integral) can be matched to a small set of frictaves and plosives. The signal determines if it is a plosive or a frictave by watching the length of the utterance (plosives occur over short periods while frictaves over long). Finally the most appropriate character is chosen.


\begin{DoxyItemize}
\item \href{http://arjo129.github.com}{\tt Return to main page} 
\end{DoxyItemize}